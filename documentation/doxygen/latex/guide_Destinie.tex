\section{Guide de prise en main du modèle Destinie}

Ce guide vise à faciliter la première prise en main du modèle, sans garantie sur la facilité de la manœuvre. La seule garantie proposée par ses rédacteurs est que dans des circonstances "utilisateurs"\footnote{Lorsque des inscriptions précises sont données, elles ont été testées sous Windows 10.}, cette procédure a fonctionné comme décrite.\\

Procédure d'installation proposée :\\
\begin{enumerate}

\item Installations préalables 
\begin{itemize}
\item Installer	R (\url{https://cran.r-project.org/}) Par exemple R version 3.5.0.
\item Installer	Rtools (\url{https://cran.r-project.org/bin/windows/Rtools/}) Choisir la version 3.5 si R version 3.5.0 ou ultérieure.
\item Disposer d'un tableur (tel Microsoft Excel ou LibreOffice Calc).
\end{itemize}




\item Installations recommandées 
\begin{itemize}
\item Installer	Rstudio (\url{https://www.rstudio.com/products/rstudio/download/}),environnement de travail R auquel il est parfois fait référence dans la documentation.
\item Installer git (\url{https://git-scm.com/})de façon à pouvoir suivre les futures mises à jour du modèle.
\item Installer par exemple GitTortoise (\url{https://tortoisegit.org/}) pour pouvoir utiliser Git en "clic-bouton".
\end{itemize}

\item Récupérer le dossier contenant le modèle : 
Créer dans C  un dossier Destinie ; puis y clôner le dépôt en le nommant destinie. \footnote{Le fait que le dossier soit dans C\\Destinie\\destinie est nécessaire pour que le modèle soit entièrement immédiatement fonctionnel.} \\
Par exemple:
\begin{itemize}
\item Faire un clic droit n’importe où
\item Cliquer sur "Git clôner"
\item  dans URL, inscrire : \url{https://github.com/InseeFr/Destinie-2}
\item dans Répertoire inscrire: \url{C://Destinie//destinie}
\end{itemize}

\item Ouvrir Rstudio et ouvrir le projet destinie.Rproj ("File" puis "Open Project")

\item Installer les packages nécessaires.
Par exemple en exécutant dans Rstudio:\\
\url{install.packages(c("devtools","Rcpp", "rstudioapi"))}\\ \url{install.packages(c("openxlsx","dplyr","tidyr","reshape2","ggplot2"))}\\  
\url{install.packages("xlsx")}\\

\item Puis charger les librairies correspondantes.
Par exemple en exécutant dans Rstudio:\\
    \url{library(devtools)} \\
    \url{library(Rcpp)} \\
    \url{library(openxlsx)} \\
    \url{library(xlsx)} \\
    \url{library(dplyr)} \\
    \url{library(tidyr)} \\
    \url{library(reshape2)} \\
    \url{library(ggplot2)} \\
\underline{Note :} au chargement de openxlsx (library(openxlsx)), une erreur peut éventuellement être rencontrée si la version de java utilisée est la version 32 bits. La version 64 bits est téléchargeable ici : \url{https://www.java.com/fr/download/manual.jsp}\\

\item Chargement du package devtools. Exécuter: \\
 \url{assignInNamespace("version_info", c(devtools:::version_info, list("3.5" = list(version_min = "3.3.0", version_max = "99.99.99", path = "bin"))), "devtools")}\footnote{Cette ligne est nécessaire au fonctionnement sous R.3.5.0 au 19 juillet 2017 ; ce passage se simplifiera dans quelques semaines, d’après : \url{https://github.com/r-lib/devtools/issues/1772\# issuecomment-388639815}}  \\
 
 \url{find_rtools( )}  \\
 Un "YES" devrait alors s'afficher.
 
 \item Création d'un fichier \url{param_perso.R} \\
Inclure dans ce fichier la ligne : \\
\url{load("test/test_RIENconfidentiel_NONrepresentatif.Rda")}\\
Il faut ici placer le chemin d'accès au fichier de données en entrée de modèle. Dans le dossier test, un jeu de données simulées, non représentatives, a été fourni afin de permettre la prise en main du modèle. La commande précédente permet de l'atteindre.\\

\item \underline{Première simulation :}\\
Ouvrir simulation.R et le faire tourner ligne à ligne.\\
A l'ouverture du fichier excel, répondre le cas échéant "oui" pour mettre à jour les liens.
\underline{Attention :} Le fichier test fourni pour prendre en main le logiciel n'est pas représentatif. Le fichier représentatif à utiliser est mis à disposition sur la plateforme Quetelet (\url{http://quetelet.progedo.fr/}).\\


\end{enumerate}

\underline{En cas de difficultés :}\\
Si vous avez des difficultés liées au modèle que vous n'arrivez pas à résoudre, sur le dépôt public (actuellement sur \url{https://github.com/InseeFr/Destinie-2}), vous trouverez un onglet : "Issues".\\
Vous pouvez rechercher si d'autres personnes ont déjà rencontré ce problème, et si la communauté des utilisateurs y a répondu, notamment en filtrant sur les étiquettes (\textit{label}) \textit{"Good for newcomers"}.\\
Si le problème persiste, vous pouvez y rédiger un message. Déclinez votre nom, puis présentez du mieux possible un exemple minimal, complet et vérifiable (voir quelques conseils sur ce sujet ici: \url{https://stackoverflow.com/help/mcve}). Enfin, étiquetez le nouveau sujet à l'aide par exemple d'un ou de plusieurs des codes suivants :
\begin{itemize}
\item \textit{"Good for newcomers"} (Bien pour les débutants)
\item \textit{"bug"}
\item \textit{"Something isn't working"} (Quelque chose ne marche pas)
\item \textit{"duplicate"} (répétition)
\item \textit{"enhancement"} (amélioration)
\item \textit{"New feature or request"} (Nouvelle fonctionnalité)
\item \textit{"This doesn't seem right"} (Cela ne semble pas correct)
\item \textit{"question"} (Question)
\item \textit{"Further information is requested"} (De plus amples informations sont demandées)\\
\end{itemize}


\underline{A noter :}\\
Au sein du projet lui-même, quelques conseils pratiques très courants sont également intégrés au sein d'un fichier \url{"conseils_pratiques.txt"}.