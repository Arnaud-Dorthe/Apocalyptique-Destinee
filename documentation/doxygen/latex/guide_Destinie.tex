\section{Guide de prise en main du modèle Destinie}
\subsection{Lancer une projection avec le modèle Destinie}
Ce guide vise à faciliter la première prise en main du modèle. La seule garantie proposée par ses rédacteurs est que dans des circonstances "utilisateurs"\footnote{Lorsque des inscriptions précises sont données, elles ont été testées sous Windows 10  sous un ordinateur où rien n'était pré-installé.}, cette procédure a fonctionné comme décrite.\\
\subsubsection{Installation minimale}
\begin{enumerate}

\item Installations préalables 
\begin{itemize}
\item Installer	R (\url{https://cran.r-project.org/}) 
\item Installer	Rtools (\url{https://cran.r-project.org/bin/windows/Rtools/}), si votre système d'exploitation est Windows pour pouvoir installer le package destinie à partir des fichiers source. Choisir la version de Rtools compatible avec la version de R choisie précédemment. 
\end{itemize}



\item Installations recommandées 
\begin{itemize}
\item Installer	Rstudio (\url{https://www.rstudio.com/products/rstudio/download/}), environnement de travail R auquel il est parfois fait référence dans la documentation.
\item Installer git (\url{https://git-scm.com/})de façon à pouvoir suivre les futures mises à jour du modèle.
\item Installer par exemple TortoiseGit (\url{https://tortoisegit.org/}) pour pouvoir utiliser Git en "clic-bouton".
\item Disposer d'un tableur (tel Microsoft Excel ou LibreOffice Calc).
\end{itemize}

\item Installer les packages nécessaires à Destinie :\\
\textit{install.packages("githubinstall"}
\item Installation du package Destinie :\\
\textit{githubinstall("Destinie-2")} (par défaut les packages dépendants seront installés)

\end{enumerate}

\subsubsection{Réaliser une première simulation}
Dans le logiciel R,
\begin{enumerate}
\item On charge le package : \textit{library(destinie)}
\item On lance une simulation exemple : \textit{demo(simulation,package="destinie,encoding="utf8")} Le fichier source \url{simulation.R} permet de lancer une simulation à partir d'un échantillon test non représentatif \footnote{\underline{Attention :} Le fichier test fourni pour prendre en main le logiciel n'est pas représentatif. Le fichier représentatif à utiliser est mis à disposition sur la plateforme Quetelet (\url{http://quetelet.progedo.fr/}).} en choisissant des hypothèses classiques.
\end{enumerate}


\subsection{Pour obtenir/utiliser les sources}

\begin{enumerate}
\item Récupérer le dossier contenant le modèle : 
Créer un dossier Destinie ; puis y clôner le dépôt en le nommant destinie. \footnote{Le fait que le dossier soit dans \url{/Destinie/destinie} pour que les liens dans les fichiers de paramètres soient corrects.} \\
Par exemple, pour les utilisateurs de TortoiseGit :
\begin{itemize}
\item Faire un clic droit dans l'explorateur
\item Cliquer sur "Git clôner"
\item  dans URL, inscrire : \url{https://github.com/InseeFr/Destinie-2}
\item dans Répertoire inscrire l'emplacement où vous souhaitez installer Destinie. \item Récupérer le dossier contenant le modèle : 
Créer un dossier Destinie ; puis y clôner le dépôt en le nommant destinie. \footnote{Le fait que le dossier soit dans \url{/Destinie/destinie} pour que les liens dans les fichiers de paramètres soient corrects.} \\
\end{itemize}
%\item Chargement du package devtools. Exécuter: \\
% \url{assignInNamespace("version_info", c(devtools:::version_info, list("3.5" = list(version_min = "3.3.0", version_max = "99.99.99", path = "bin"))), "devtools")}\footnote{Cette ligne est nécessaire au fonctionnement sous R.3.5.0 au 19 juillet 2017 ; ce passage se simplifiera dans quelques semaines, d’après : \url{https://github.com/r-lib/devtools/issues/1772\# issuecomment-388639815}}  \\
% 
% \url{find_rtools( )}  \\
% Un "TRUE" devrait alors s'afficher.
 
 \item  Installer le package devtools, et dans Rstudio utiliser les commandes de Build pour charger destinie ou compiler le package source .tar.gz ou le windows binairies. 
\end{enumerate}



\underline{En cas de difficultés :}\\
Si vous avez des difficultés liées au modèle que vous n'arrivez pas à résoudre, sur le dépôt public (actuellement sur \url{https://github.com/InseeFr/Destinie-2}), vous trouverez un onglet : "Issues".\\
Vous pouvez rechercher si d'autres personnes ont déjà rencontré ce problème, et si la communauté des utilisateurs y a répondu, notamment en filtrant sur les étiquettes (\textit{label}) \textit{"Good for newcomers"}.\\
Si le problème persiste, vous pouvez y rédiger un message. Déclinez votre nom, puis présentez du mieux possible un exemple minimal, complet et vérifiable (voir quelques conseils sur ce sujet ici: \url{https://stackoverflow.com/help/mcve}). Enfin, étiquetez le nouveau sujet à l'aide par exemple d'un ou de plusieurs des codes suivants :
\begin{itemize}
\item \textit{"Good for newcomers"} (Bien pour les débutants)
\item \textit{"bug"}
\item \textit{"Something isn't working"} (Quelque chose ne marche pas)
\item \textit{"duplicate"} (répétition)
\item \textit{"enhancement"} (amélioration)
\item \textit{"New feature or request"} (Nouvelle fonctionnalité)
\item \textit{"This doesn't seem right"} (Cela ne semble pas correct)
\item \textit{"question"} (Question)
\item \textit{"Further information is requested"} (De plus amples informations sont demandées)\\
\end{itemize}


\underline{A noter :}\\
Au sein du projet lui-même, quelques conseils pratiques très courants sont également intégrés au sein d'un fichier \url{"conseils_pratiques.txt"}.
